\documentclass[cjk]{beamer}
\usepackage[UTF8,noindent]{ctexcap}
% Warsaw
% PaloAlto
% AnnArbor
% Malmoe
\usepackage{fontspec,xunicode}
%\defaultfontfeatures{Mapping=tex-text}
\usetheme{Boadilla}
\useinnertheme{circles}
% \useoutertheme[height=0\textwidth,width=0.18\textwidth,hideothersubsections]{sidebar}
% \useoutertheme{default}
\usecolortheme{whale}

% \newfontfamily\kaisu{STKaiti}       %定义华文楷体为\kaisu
% \setsansfont{TeX Gyre Termes}       %设置西文字体为times new roman
% \setCJKsansfont{SimSun}             %设置中文字体为宋体
% \setCJKmonofont{STKaiti}
% \setmonofont{TeX Gyre Termes}   
\setbeamerfont{frametitle}{family=\ttfamily}

\title{计算方法}
%\subtitle{\textsc{Beamer}}
\author{张世琛}
\institute{计科1802\\ 计算机科学与技术学院}
\date{\today}
\begin{document}

\frame{\titlepage}

\begin{frame}{目录}
    \tableofcontents
\end{frame}

\begin{frame}{帧标题1}{子帧标题1}
    \section{帧标题1}
    \subsection{子帧标题1} 

    \begin{block}{这是一个公式}
        $$f(x)=\sum_{x=1}^N{\dfrac{1}{e^x(x+1)}}$$
    \end{block}
    在这里输入你要写的内容
\end{frame}

\begin{frame}{帧标题2}
    \section{帧标题2}
    \begin{proof}
        这里写你的证明过程
    \end{proof}
    在这里输入你要写的内容+1
\end{frame}

\begin{frame}{帧标题3}
    \section{帧标题3}
    在这里输入你要写的内容+1\par
    \begin{enumerate}
        \item<1->列出你的条目1
        \item<2->列出你的条目2
    \end{enumerate}
\end{frame}

\end{document}